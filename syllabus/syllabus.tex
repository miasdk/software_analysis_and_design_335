\documentclass[a4paper]{article}

\usepackage{fullpage} % Package to use full page
\usepackage{parskip} % Package to tweak paragraph skipping
\usepackage{tikz} % Package for drawing
\usepackage{amsmath}
\usepackage{hyperref}

\title{CSCI 33500 Software Analysis and Design 3}
\author{Instructor: Jaime Canizales}
\date{Monday - Thursday, West Bldg W604, Summer 2025}

\begin{document}
\maketitle

\textbf{Textbook:} Data Structures \& Algorithm Analysis in C++, 4th Edition. Mark Allen Weiss. Pearson, New
York. ISBN: 013284737x.
\\\\
\textbf{Email:} jaime.canizales@hunter.cuny.edu \\
\textbf{Office Hours:} By appointment.

\textbf{Course Description:}\\ The principal objective of this course is to further your understanding of the design and analysis of
algorithms and data structures. This includes the introduction of new abstract data types, including
hashes, heaps, various forms of trees, graphs, and the sorting problem from a higher perspective than
was [supposed to be] presented in CSci 235. It also covers worst and average case behavior analysis
and optimality, and to a much smaller extent, polynomial time complexity classes and theory. Another
objective of the course is to develop your C++ programming and software engineering skills a little more.
This course requires that you write more complex software than you have done in the preceding courses.

\textbf{Grading: } \\
Midterm Exam: \hspace{1mm} 40\% \\
Final Exam: \hspace{6mm} 40\% \\
Project: \hspace{13mm} 20\%
\\\\
\textbf{Exams:} There will be one midterm and a final. The first midterm will cover the first five chapters of
the textbook. The final will cover chapters six to nine. These chapters are not covered in their entirety,
check notes to reviews topics covered from chapters.\\

\textbf{Projects: } There will be three assignments, which will be posted on blackboard.

\section*{Hunter College Official Policies:}
\subsection{Academic Integrity Statement:} 
“Hunter College regards acts of academic dishonesty (e.g., plagiarism, cheating
on examinations, obtaining unfair advantage, and falsification of records and 
official documents) as serious offenses against the values of intellectual 
honesty. The College is committed to enforcing the CUNY Policy on Academic 
Integrity and will pursue cases of academic dishonesty according to the Hunter
College Academic Integrity Procedures.”  
URL: https://hunter.cuny.edu/center-for-online-learning/online-learning-resources/academic-integrity-policy/

Please consider adding the following statement  from the policy as well:
\begin{itemize}
    \item Copying from another person or from a generative AI system or allowing
    others to copy work submitted for credit or a grade. This includes uploading
    work or submitting class assignments or exams to third party platforms and
    websites beyond those assigned for the class, such as commercial homework
    aggregators, without the proper authorization of a professor. Any use of 
    generative AI tools must be in line with the usage policy for specific 
    assignments as defined in the course of the syllabus and/or communicated by
    the course instructor.
    \item Using artificial intelligence tools to generate content for assignments
    or exams, including but not limited to language models or code generators, 
    without written authorization from the instructor
\end{itemize}

\subsection{Anti-harassment statement:}
Bullying, cyberbullying, online hate, intimidation, threats, harassment, and 
pressure to share schoolwork are all forms of violence. CUNY holds a zero 
tolerance stance towards all such acts. The University is committed to prevention
of any form of bullying, will respond promptly to threats and/or acts, and will 
protect victims of bullying from retaliation. As a criminal matter, the New York
Attorney General defines cyberbullying as the use of email, websites, instant 
messaging, chat rooms, text messaging and digital cameras to antagonize and 
intimidate others. Disrupting a teleconferencing platform 
(such as Zoom/Skype/Blackboard Collaborate Ultra) is a federal crime.


\subsection{Hunter College Policy on Sexual Misconduct:}
“In compliance with the CUNY Policy on Sexual Misconduct, Hunter College reaffirms
the prohibition of any sexual misconduct, which includes sexual violence, sexual 
harassment, and gender-based harassment retaliation against students, employees,
or visitors, as well as certain intimate relationships. Students who have 
experienced any form of sexual violence on or off campus 
(including CUNY-sponsored trips and events) are entitled to the rights outlined 
in the Bill of Rights for Hunter College.
\begin{itemize}
\item Sexual Violence: Students are strongly encouraged to immediately report the
incident by calling 911, contacting NYPD Special Victims Division Hotline 
(646-610-7272) or their local police precinct, or contacting the College's Public 
Safety Office (212-772-4444).
\item All Other Forms of Sexual Misconduct: Students are also encouraged to contact
the College's Title IX Campus Coordinator, 
Dean John Rose (jtrose@hunter.cuny.edu or 212-650-3262) or 
Colleen Barry (colleen.barry@hunter.cuny.edu or 212-772-4534) and seek complimentary 
services through the Counseling and Wellness Services Office, Hunter East 1123. 
CUNY Policy on Sexual Misconduct 
Link: http://www.cuny.edu/about/administration/offices/la/Policy-on-Sexual-Misconduct-12-1-14-with-links.pdf”
\end{itemize}


\subsection{ADA Statement:}
“In compliance with the ADA and with Section 504 of the Rehabilitation Act, Hunter 
College is committed to ensuring educational access and accommodations for all its 
registered students. Hunter College's students with disabilities and medical 
conditions are encouraged to register with the Office of AccessABILITY for 
assistance and accommodation. For information and appointment contact the Office of 
AccessABILITY located in Room E1214 or call (212) 772-4857 /or VRS (646) 755-3129.”

\subsection{Change Policy:}
Sample Language: "Except for changes that substantially affect implementation of 
the evaluation (grading) statement, this syllabus is a guide for the course and
is subject to change with advance notice." (Alternate use: "Note that details of
this document are subject to change if the need arises.")

Wish you all a great start to the semester.
\end{document}